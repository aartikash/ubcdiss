\chapter{Ripple Attacks}
\label{attack}

\section{Motivating system: Artificial Pancreas System}


We use a DNN based \ac{APS}, closed-loop model, by Dutta et al. \cite{10.1007/978-3-319-99429-1_11}  as our motivating example for illustrating a \ac{RFDIA}. 
We use this to explain our motivating attack. 
The \ac{APS} model we use in this section is also our first evaluation system; it is also the simple \karthik{simplest of the three systems ?} in terms of \ac{DNN} complexity described in ~\ref{dnncomplexity}.
A patient relies on  \ac{APS} to correctly determine the next dose of insulin to be injected every $t$ minutes. 

\begin{figure}
	\centering
	\includegraphics[width=0.7\linewidth, height=0.3\linewidth]{Images/APSDNN}
	\caption[APS DNN]{APS DNN designed by Dutta et. takes in 74 inputs of insulin and glucose. The next layers form connections between the insulin and the glucose to make predictions.[8]}
	\label{fig:apsdnn}
\end{figure}

%How is APS constructed?
The APS architecture is explained in Section ~\ref{apsdnn}. 
We examine the \ac{APS} controller model from Dutta et al., which has a feed-forward architecture. 
The DNN for APS creates mappings between the insulin and the glucose values that allow for the prediction of future insulin values as shown in Figure ~\ref{fig:apsdnn}. 
The insulin and the glucose values are the values collected from the sensors and the actuators. 
The model has 74 inputs in total, where 32 inputs are the glucose, and the other 32  are insulin values collected every 5 minutes.
 The DNN layers use these values as inputs to predict the next value. 
 This implies that the future value is predicted based on the inputs from the previous set of values. 


%Defense mechanisms
There are two phases in a \ac{DNN}s life; training phase and inference phase. 
During the training phase, there are three \ac{DNN}s that are trained in an \ac{APS} to predict the next outcome. 
The first \ac{DNN} is used to predict the next outcome. 
The accompanying two \ac{DNN}s are for the purpose of designing defense mechanisms such that the patient is not overdosed with insulin.  
The two \ac{DNN}s are separate from its main decision making controller. 
These \ac{DNN}s are trained based on the latest 30-day patient data which is collected from monitoring the patient, to understand the standard injections for a patient over time. 

One \ac{DNN} learns a lower threshold on injection amount; the other learns an upper bound on injection amount. 
Hence, for every time of the day based on the previous patient characteristic, the minimum and the maximum values are known. 
However, the system will not detect  an adversary  who manages to always administer the maximum allowed dosage through \ac{RFDIA}. 
Hence, the attacker needs to identify inputs and  perturb them  such that the output changes the maximum allowed dosage for every injection,
  while being lower than the upper threshold to avoid detection, and cause maximum potential damage to the patient. 


\section{Attack Model}
The attacker's goal is to manipulate the sensor measurements to conduct FDIAs without triggering alarms as shown in Figure ~\ref{fig:attackmodelphysical}. 
The attacker can use network noise or physical means of sensor tampering to attack the systems ~\cite{10.1145/3319535.3339815}.
 
\begin{figure}
	\centering
	\includegraphics[width=0.7\linewidth]{Images/Attackmodelphysical}
	\caption{Attack model}
	\label{fig:attackmodelphysical}
\end{figure}

We assume that the attacker has following capabilities:
\begin{enumerate}
	\item The \ac{DNN}  architecture  is known to the attacker for example., a feed forward  network in case of an \ac{APS}. This information easy to find, as the architectures are  usually public information for the systems. 
	\item  The weights and bias of the \ac{DNN}  architecture are known as well through a read-only access to the system.  
	\item The attacker cannot modify the code of the system, but can modify the inputs to the model.
	\item The \ac{DNN} contains ReLU as its activation function. %All three systems in this work contain ReLU.
\end{enumerate}

\subsection{Strawman attacker}
A simple way to attack the system would be to change all the 74 values that are the inputs to the DNN.
 If all the inputs are changed, the final output prediction is going to be wrong. The problem in this particular scenario is that 
 these inputs are collected every five minutes from the sensors attached to the patient. 
 This means that the attacker will have to conduct FDIAs every five minutes when the data is being collected, to cause a change in the output. 
 However, this is quite tedious in a practical scenario and hence, the attacker needs a better way of attacking the system. 

\subsection{Sophisticated Attacker}
A more sophisticated approach to attack the system would be by perturbing one or two inputs out of the 74 inputs that cause a change in the output. There are two ways to proceed when the attacker tries to change just one or two inputs at a time. 

\subsubsection{Attack 1}
The attacker can randomly choose two inputs and perturb them by huge amounts. 
This will indeed cause a wrong output prediction. 
However, if the input is perturbed by large amount, the error detection mechanisms will recognize that there is an anomaly. 
To prevent this the attacker can choose to perturb the two inputs by small amounts. 
However, perturbing any two random inputs by very small amounts might not necessarily lead to a wrong output prediction as shown in Chapter ~\ref{evaluation}. 

\subsubsection{Attack 2}
Adding one more layer of sophistication, the attacker does not
know the critical inputs or the input whose perturbation can lead to wrong predictions.
 This will be a more targeted approach for attacking the system since the input selection will not be random. 
However, not knowing the precise amounts by which the critical inputs should be perturbed can lead to an unsuccessful attack. 
If the perturbation is too high, it will trigger the error-detection mechanisms, 
and if it is too low, it will not affect the output as shown in Chapter ~\ref{evaluation}. 

\subsubsection{Attack 3}
The most sophisticated means for an attacker to attack the system would be to know precisely which inputs to perturb and by what amounts,
 such that the output prediction is wrong. 
To do so, the attacker needs an advanced\karthik{automated?} technique that in limited time, finds the critical inputs
 and also the small perturbations \karthik{smallest possible?} that can ensure a wrong output prediction as explained in \karthik{??}. 

%Our motivation for designing the technique were these different ways of attacking the system that can automatically synthesize attacks in an efficient way. 

\chapter{Challenges}
We face  the following challenges during the design of the technique. 
The  first challenge is choosing a mathematical abstraction technique that works best for our use case. 
The second challenge is mapping the \ac{DNN}  model to the abstraction. 
Finally, the last challenge is designing the \ac{DNN}  for our use case which is attack synthesis. 


\section{Selecting  the automated technique}
The first part is choosing a technique that allows the attacker to model a DNN and find the critical inputs through the modeling. 
There are multiple means of modeling the DNN to synthesize attacks some of which are using SMT solvers, symbolic execution and MILP.
Our goal is twofold: 
\begin{enumerate}
	\item Generality: \ac{APS} has an architecture that is despite being similar to our other evaluated systems is quite different from air control traffic management systems (\ac{ACAS-Xu} and \ac{HCAS}) as observed from Table 6.1.
	This is due to the difference in the application domain and the \ac{DNN} architecture design. 
	\ac{APS} is a medical system that consists of a standard feed-forward network with no exotic features such as normalization in the network unlike \ac{ACAS-Xu}
	Hence, we want an abstraction technique that supports such different architectures. 
	\item Scalability: \ac{APS} consists of 2 hidden layers with 50 neurons each which basically means that there are $2800$ ($50x5 + 50x50 + 50x1$) connections in total.
	\ac{ACAS-Xu} consists of 5 hidden layers with 50 neurons each which means the total connections are $10,500$. 
	Hence, we need a technique that is scalable to bigger systems. 
	
\end{enumerate}

 We show that \ac{MILP} fulfills these two criteria by providing us generality and scalability. 


\section{ Mapping  Mixed-Integer Linear Programming Model to Deep Neural Networks}

DNN architecture is quite complex and researchers have used experimental means to understand the security scenarios in the DNN. 
Their  mathematical structure contains multiple layers and the non-linearity introduced due to the activation functions within the network add another layer of complication for modeling in \ac{MILP}.
There are two main complications that arise here:
\begin{enumerate}
	\item Modeling each layer as a set of constraints to build a \ac{MILP} model.
	\ac{APS} has two layers as explained in Chapter 3 and has 50 neurons each. 
	We use that to 
	\item Non-linear terms are not allowed in \ac{MILP} modeling. Hence the big question here is how can we model the activation functions. 
\end{enumerate}

We formalize the \ac{DNN} to represent the entire network as a set of linear constraints. 
We tackle the non-linearity of activation functions by defining them as piece-wise linear which we demonstrate for one activation function ReLU.
We chose ReLU because are three evaluation systems have ReLU in their models. 

\section{Modeling cost function}
The goal of the cost function is to minimize or maximize the linear constraints that represents the \ac{DNN} architecture. 
However, our requirement is different in the traditional sense of using minimization and maximization which is usually used in terms of resource utilization. 
Based on our attack model we want to minimize the input perturbations and maximize the output deviations in a \ac{DNN}.
We do so to find the optimal solutions which we have defined in Section 2.5
There are two main problems that can arise based on our cost function modeling:
\begin{enumerate}
	\item There is a possibility that there are no solutions. 
	This happens if the \ac{MIP} is infeasible and we need to redefine our cost function. 
	\item If the \ac{MIP} is feasible, then there is a possibility that there is no optimal solution since there are infinitely many good cost function values. 
	This makes our solution space unbounded and can lead to state space explosions. 
\end{enumerate}

We demonstrate both the problems in our experimentation and show how we tackle the state space problem that can arise thereby scaling our approach for big sized systems. 
We use interval analysis which we elaborate in Chapter 5.
































