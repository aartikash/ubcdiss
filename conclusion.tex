\chapter{Conclusion}
\label{conclusion}


Based on our results, we discuss two approaches that can help researchers and developers design error-free \ac{DNN}s. 
We also discuss some limitations of our approach.
Finally, we conclude, and discuss potential future work.


\section{Designing Robust DNNs}
From our experience in attacking three different types of \ac{DNN} based CPS, there are two ways to construct or build robust \ac{DNN} based systems.
We based this on the following observations. \karthik{I'm not sure if this is the correct term as you don't really say what the observation was.} 
%The two things that help us in building our intuition were:
\begin{enumerate}
	\item Building and running the models on \ac{MILP} solvers helped us understand the power in terms of reasoning about \ac{DNN} based systems for observing non-obvious security properties such as ripples. \karthik{Ripple is a security property ? What do you mean "help  you understand" ? Did you mean showed the feasibility of ?}
	\item Attacking three practical systems helped  us come to a realization how to design robust \ac{DNN}  by adding layers of complexity to \ac{DNN}. \karthik{Can you be more specific about how you came to this realization exactly ?}
\end{enumerate}


\subsection{Effects of normalization and attacks}
The biggest difference between \ac{APS} and \ac{ACAS-Xu} apart from the \ac{DNN} size was the robustness of the \ac{DNN}. 
\ac{APS} has a fully-connected feed-forward architecture, where the network takes the inputs and passes them through the hidden layers to compute the output. 
\ac{ACAS-Xu} is also a fully-connected \ac{DNN}; however, it includes normalization.
This means that we have to try out more combinations of input variable perturbations to find if an attack exists. \karthik{We = \tool ?}
Small perturbations during the \ac{FDIA} get masked due to the normalization layer; changing an input value from $5$ to $10$ has little to no effect on the \ac{DNN} if the normalization layer normalizes a range of inputs to a value between $0$ and $1$. 
Thus, we have to perturb \karthik{the inputs ?} by values large enough to induce a change in the normalized representation. 
We also observed that in many cases, despite large enough perturbation, we were unable to generate \ac{RFDIA} in \ac{HCAS} and \ac{ACAS-Xu} as shown in Chapter ~\ref{evaluation}.
Intuitively, based on the sizes of \ac{APS} and \ac{ACAS-Xu} we should have been able to generate significantly more \ac{RFDIA} \karthik{than APS?}; instead we observe that despite the bigger search space in \ac{HCAS} and \ac{ACAS-Xu} we found very few \ac{RFDIA} as compared to \ac{APS}. 

Therefore, we believe that normalization makes it difficult for the attacker to compromise the system \karthik{via RFDIAs}. 
It does not eliminate the possibility of conducting attacks entirely, but makes the system somewhat more secure. \karthik{This is a wishy-washy statement; it's either more secure or it is n't. Also, secure against what ?} 

\subsection{Effects of DNN size and attacks}
The \ac{APS} was a relatively smaller system compared to \ac{ACAS-Xu} and \ac{HCAS}.
We observe that due to the smaller and much less complex\karthik{if you define complexity in terms of size, these are synonymous, aren't they ?} nature of \ac{APS}, we are able to try out more combinations and hence are able to find more attacks. \karthik{Agai, is we == \tool ?}
Every attack in \ac{APS} takes less than a minute to generate if it exists, or to tell us no attacks are possible; 
\ac{HCAS} and \ac{ACAS-Xu}, on the other hand can take hours, just to tell us that there is no attack. \karthik{Or to find attacks if they do exist ?} 
This restricts and allows us to conduct a limited number of combinations to find attacks. \karthik{Restricts to what?}
Hence, comparing between small systems such as \ac{APS}, and relatively bigger systems, such as \ac{HCAS} and \ac{ACAS-Xu}, we observe that the small system size enables us to generate more attacks and cover more combinations. \karthik{I think this is a rather trite observation, and doesn't need to be repeated.}
When we compare \ac{HCAS} and \ac{ACAS-Xu} we observe a similar behavior; \ac{HCAS} took less time per attack and hence we were able to explore more space and find successful instances of attacks. 
The reason we consider time an important factor is because if it takes 100 days to find an attack, attacks become more expensive to find. 
Hence, our focus is not to generate the attack in seconds but our goal is to come to a reasonable trade-off between time and size to generate attacks. 

Thus, based on our observations we can say that designing bigger systems can help make systems more secure but this is true given they are well trained since if a network is poorly trained no matter how big it is will not help secure it. \karthik{So far you never mentioned training ? Why is this coming now ? I suggest removing this.}




\subsection{Summary}
Thus keeping the above two principles in mind, adding normalization and having certain level of complexity in the \ac{DNN} can significantly help the \ac{DNN}s be more secure. \karthik{did you mean increasing the complexity ? Certain level can be small or even 0 !} 
This will obviously not work in every situation because the \ac{DNN} might have other flaws but it is good to keep this in mind when designing networks especially for safety-critical systems. \karthik{I'd have thought safety-critical systems would require hard guarantees !}

\section{Debugging existing DNN using \tool}
In our work, we used\karthik{actually, we proposed it !} \tool to generate attacks for trained \ac{DNN}s that also tell us if in a particular setting, no attacks exist. 
However, instead of trying to find attacks in the \ac{DNN}, we can also use \tool for debugging; i
t means that we can conduct an input-output pair generation to study if there are some abnormal pairs that are being generated. \karthik{What's an abnormal pair in this context ?}
This can be seen as trying to generate the datasets from a trained model that can be used to understand the corner cases; cases can help the designers or researchers to understand how to modify the training data such that they can improve a \ac{DNN} performance by adding or removing from the datasets. \karthik{I don't understand this - perhaps an example would help}



\section{ Limitations  and Future Work}

There are 2 limitations of \tool that pave way for future work. 
\begin{enumerate}
	\item First, we are assuming that the attacker has read access to the weights and the bias of a \ac{DNN}.
	However, in an ideal scenario it is difficult to obtain read access to weights and bias of safety-critical systems such as \ac{CA} systems since they protect their systems by security by obscurity for an attacker. 
	Therefore, the attacker might need tools or techniques that allows them to attack the system without having knowledge about the internals of the network; also called as a black-box approach. 
	This would mean that with a short amount (preferably a couple of hours) the attacker is able to attack a black box system and conduct \ac{RFDIA} or other attacks. 
	\item Second, we observed that in complex \ac{DNN}s of our three systems, there were many cases when  \ac{ACAS-Xu} timed out and did not return us any results. 
	The time out occurred because the search space for the appropriate input-output pair was very huge. 
	However, we believe adding more precise bounds and limiting the search space can significantly scale our approach. 
	There is a possibility of designing clever heuristics for domain specific systems such as removing non required layers, pruning the \ac{DNN} appropriately, using algorithmic approaches to decrease time complexity of \ac{MILP} solvers. 
	
	\label{section:limitations}
	
\end{enumerate}

\karthik{Can we cleanly separate extensions from limitations in a separate section ? I think the debugging paragraph should also be an extension, not a separate section by itself.}

Finally, we believe that there are many more extensions that can be done with \tool. 
\begin{enumerate}
	\item First, it can be extended for multiple different application domains to find attacks.
	For instance \ac{AV} \karthik{Iguess you mean the terrestrial kind, as you've already considered UAVs} have multiple stacks of \ac{DNN}s to compute the output, and \tool can be utilized to find attacks in multiple \ac{DNN}s. \karthik{Perhaps say how this is different from what you did}
	\item Second, one can model cost functions for an attack model different from ours and use our insights to generate attacks. \karthik{Can you give an example ?}
	This is because the modeling of the \ac{DNN} and \tool can be easily reused by integrating with different attack specific cost functions. 
\end{enumerate}


\section{Conclusion}
\karthik{Recap the broad problem first.}

We demonstrate \tool, which is based on generating \attack using backend \ac{MILP} solvers. 

We address the two main challenges in synthesizing \attack: 
\begin{enumerate}
	\item Locating the critical inputs: 
	
	\item Finding the smallest perturbations for critical inputs to conduct \attack.
	
\end{enumerate}
To locate the critical inputs, we model a cost function and add $delta$ variable to the inputs and minimize the $delta$. 
This tells the attacker changing the minimal set of inputs by smallest possible amounts will result in wrong output predictions.\karthik{Not sure what it tells the attacker ? Did you mean it tells him/her what to change and how - please check your grammar.}

Based on our experiments, we conclude that:

\begin{enumerate}
	\item The first observation is that it is easier (in terms of time)\karthik{just say faster!} to find attacks on small-sized DNN based CPS. 
	APS took less than a second for finding an attack whereas ACAS Xu took as high as 5 hours and timed out for certain constraints. 
	
	\item We found fewer attacks in ACAS Xu as compared to APS indicating  that ACAS Xu is a more robustly designed network due to the existence of normalization that makes it difficult to generate ripples. 
	Hence, it is possible to design robust \ac{DNN} while designing the architectures by integrating such features. \karthik{I think this sentence is more speculation than observation.} 
	
	\item Finally, we observe that the choice of the cost function and the intervals affects the state explosion of the system. \karthik{State explosion or state-space explosion ?}
	If there is a minimization function on the input perturbations, it takes much longer to compute and can also lead to state-space \karthik{explosion ?} in certain cases as comparing with other function.  \karthik{I don't understand where you found this earlier. Is this a new observation ?}
\end{enumerate}




\iffalse
We have addressed security as an optimization problem by modeling the Deep Neural Network as a Mixed Integer Linear model. This is the first step to automatically synthesize attacks called ripple attacks that on small perturbations to the input propagate further to cause output perturbations. 

We discussed the modeling details, designing attack specific cost functions \smi{change 'cost functions' to 'target inputs/perturbation bounds'; cost functions mean objective function which doesn't change for us}, finding critical inputs and finally finding the minimum perturbations for a successful attack that we call as a ripple attack. We show our evaluation on three systems of different sizes. Our biggest system size is (5,50,50,50,50,50,5). We model the cost functions such that even for big systems, our approach does not blow up \smi{we should chat and clarify the meaning of cost functions in this context}. We chose the three systems that are practical safety-critical systems that is an artificial pancreas system, and two aircraft collision avoidance networks.

For the medical system which is a considerably smaller system than our collision avoidance system, we are able to find the critical inputs and synthesize minimum perturbations in less than a second. For bigger system with more layers and number of neurons per layer, we are successfully able to synthesize attacks within a minute. 

We believe that using our technique to find similar attacks in systems such as self-driving cars that have multiple DNNs instead of one in their control system would be an interesting future work. As mentioned in the discussion section using our approach for tracing error propagation is another interesting use case. Finally, modeling application specific heuristics for systems such as surgical arm to synthesize attacks is another interesting topic for future research. 
\fi 
%\chapter{Discussion}
In this section we discuss how to model our insights into modeling robhust \karthik{Spell Check Fail!} 
DNNs by two approaches. \karthik{Did you mean leverage insights ? Also, what're the approaches for ?} 
We also discuss some limitations in our approach.

\section{Designing Robust DNNs}
From our experience in working with three different types of DNN based CPS to attack the systems, we realized that there are two ways to construct or build robust DNN based systems. 
\karthik{Can you say a little bit about how you came to this realization ? Was it based on the experimental results ?}

\section{Designing Robust DNN from scratch}
The biggest difference between APS and ACAS Xu apart from the DNN size was the robustness of the DNN. APS had a simple feed-forward architecture, where the network took the inputs, applied the non-linearity to the equations \karthik{non-linearity is not an action !},
 and calculated the outputs. Whereas ACAS Xu applied normalization to its inputs. \karthik{This is a phrase, not a sentence}

APS had accompanying DNNs that kept\karthik{Duh ?} did a range check to ensure that the output prediction lies within certain bounds \karthik{Did you mean specified bounds ?}. 
This was easier to break as per our attack model as compared to ACAS Xu\karthik{Can we quantify easier? Also, by break, I guess you mean attack?}. 
In ACAS Xu, due to the normalization of the inputs, the perturbations had to be carefully designed since one input perturbation did not easily perturb the outputs easily \karthik{Too many easilys!}. 
We can also observe the percentage of successful attacks from Table II \karthik{Symbolic ref.?} in APS and ACAS Xu. \karthik{What did you observed about them?}
Hence, we believe that applying techniques while training and modeling DNN architecture can significantly help prevent \attack. \karthik{This is a vacuous statement. What techniques are these ? How does one go about designing such a DNN?}

\section{Debugging existing DNN using \tool}
Given there exists a system with DNN similar to that of APS without inbuilt mechanisms, we believe that our tool can help in debugging the DNN by constructing a comprehensive list of cost functions and running the model for different scenarios. This might not provide complete coverage but can certainly help in the falsification process. 
\karthik{So what's falsification here. Also, it seems like an overly restrictive statement to make about the similarity with APR. What's inbuilt mechanism here ?}

\section{ Limitations}

There are 2 limitations in our work.
First, we are assuming that the attacker has access\karthik{Read or write ? Be specific !} to the weights and the bias. However, in a more ideal attack scenario \karthik{for the attacker ?}, if we \karthik{Who's we?} can find a way to  attack the system without knowing the weights and bias of the system, that would be really interesting \karthik{Please avoid weasel words like interesting - make it clear what is of interest}. %Figuring out FDI attacks in that scenario would be challenging yet rewarding. 
Second, for bigger systems such as ACAS Xu, we observed that running all combinations was taking a lot of time (approx 13 hours), only to tell us that no attack exists \karthik{I suppose us is we as the researcher here. Also, is the time taken a function of the attacks or the network ?}. We think there is scope of improvement to provide speedups by applying clever heuristics to the model. \karthik{Perhaps give 1-2 examples of heuristics here.}


%The limitation of our work is that we do not evaluate the completeness of our approach. We show experimentally that our technique always provides us a solution if the model is properly modeled and if there are attacks possible. Otherwise, it returns an infeasible model.  Since the modeling of the network is represented as a set of well-defined equations it is always going to return a solution if it exists. If no solution exists then it will return the model is infeasible. 
%The interesting part of our approach is that we are not trying to verify but instead we are trying to find feasible FDI attacks for the system. 
\label{section:limitations}