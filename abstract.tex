%% The following is a directive for TeXShop to indicate the main file
%%!TEX root = diss.tex

\chapter{Abstract}

\ac{CPS} are deployed in many mission-critical applications such as medical devices (e.g., an \ac{APS}), autonomous vehicular systems (e.g., self-driving cars, unmanned aerial vehicles), and aircraft control management systems (e.g.,  \ac{HCAS} and \ac{ACAS-Xu}). 
Ensuring correctness is becoming more difficult as these systems adopt new technology, such as \ac{DNN}, to control these systems. 
\ac{DNN} are black box algorithms whose inner workings are complex and difficult to discern due to their data-driven nature.
As such, understanding their vulnerabilities is also complex and difficult. 


We provide a systematic approach to synthesize a new category of attacks called \ac{RFDIA} in \ac{DNN} based mission-critical systems.
\ac{RFDIA} is carried out by perturbing specific inputs that we call critical inputs by minimal amounts to  change the \ac{DNN}'s output stealthily. These  perturbations propagate as ripples over multiple \ac{DNN} layers and can lead to corruptions that  can be fatal in certain cases. 
We demonstrate that it is possible  to construct such attacks efficiently by identifying the \ac{DNN}'s critical inputs. 
The critical inputs are those that  affect the final outputs the most on being perturbed. 
Understanding this new class of attacks sets the stage for developing methods to mitigate these vulnerabilities. 

Our attack synthesis technique is based on modeling the attack as an optimization problem using \ac{MILP}.
We define an abstraction for \ac{DNN} based \ac{CPS} that allows us to automatically: 1) identify the critical inputs, and 2) find the smallest perturbations. 
We demonstrate our technique on three practical \ac{CPS} with two mission-critical applications in increasing order of complexity: Medical systems (\ac{APS}) and aircraft control management systems (\ac{HCAS} and \ac{ACAS-Xu}). 
Our key observations for scaling our technique to complex systems such as \ac{ACAS-Xu} were to define: 1) appropriate intervals for their inputs and the outputs, and 2) attack specific objective (cost) functions in the abstraction.  
 



% Consider placing version information if you circulate multiple drafts
%\vfill
%\begin{center}
%\begin{sf}
%\fbox{Revision: \today}
%\end{sf}
%\end{center}
