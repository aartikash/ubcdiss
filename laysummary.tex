%% The following is a directive for TeXShop to indicate the main file
%%!TEX root = diss.tex

%% https://www.grad.ubc.ca/current-students/dissertation-thesis-preparation/preliminary-pages
%% 
%% LAY SUMMARY Effective May 2017, all theses and dissertations must
%% include a lay summary.  The lay or public summary explains the key
%% goals and contributions of the research/scholarly work in terms that
%% can be understood by the general public. It must not exceed 150
%% words in length.

\chapter{Lay Summary}

\ac{CPS} are systems that perform computations by continuous interactions with the physical environment and perform actions in the physical environment based on the computations. 
\ac{CPS} are deployed in many mission-critical applications such as medical devices (e.g., an \ac{APS}), autonomous vehicular systems (e.g., self-driving cars, unmanned aerial vehicles), and aircraft control management systems (e.g.,  \ac{HCAS} and \ac{ACAS-Xu}). 
This mission-critical aspect of CPS suggests that ensuring their correct functioning, and complying with all safety specifications is of paramount importance. 
Ensuring correctness is becoming more difficult as these systems adopt new technology, such as Deep Neural Networks, to control them. 
\ac{DNN} are black box algorithms whose inner workings are complex and difficult to discern.
As such, understanding their vulnerabilities is also difficult. 

This work addresses the problem of understanding vulnerabilities by introducing a  technique to produce attacks that can alter the behavior of the system in unexpected ways. 
We demonstrate that it is possible  to construct such attacks efficiently. 
Understanding this new class of attacks is the first step towards developing methods to mitigate these vulnerabilities. 



 


%Basically three things are being done here
%1- new category of attacks
%2- formalised them
%3- showed them on three different systems

