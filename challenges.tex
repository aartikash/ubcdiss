\section{Challenges}
There are three challenges in the design of this technique. 
(1) choosing a mathematical abstraction that works best for our use case, 
(2) mapping the \ac{DNN}  model to the abstraction, and 
(3) designing the \ac{DNN}  for attack synthesis.  


\subsection{Selecting  the automated technique}
The first part is choosing an abstraction that allows the attacker to model a DNN and find critical inputs. 
There are multiple possible approaches such as \ac*{SMT} solvers, symbolic execution and \ac*{MILP}.

We desire a model that is
\begin{enumerate}
	\item General: \ac{APS} has an architecture that %despite being similar to our other evaluated systems 
	is quite different from that of an air control traffic management system (\ac{ACAS-Xu} and \ac{HCAS}) as observed from Table 6.1.
	This is due to the difference in the application domain and the \ac{DNN} architecture design. 
	\ac{APS} is a medical system that consists of a standard feed-forward network without normalization; \ac{ACAS-Xu} normalizes its inputs by scaling them to a range of {0-1}, which makes it difficult to perturb the inputs. 
	To accommodate normalization, we need additional details in our abstraction.  
	Hence, we want an abstraction that supports different architectures. 
	\item Scalable: The fully-connected \ac{APS} \ac{DNN} consists of $2$ hidden layers with $50$ neurons each which basically means that there are $2800$ ($50*5 + 50*50 + 50*1$) connections in total.
	On the other hand, \ac{ACAS-Xu} consists of 5 hidden layers with 50 neurons each, for $10,500$ total connections. 
	Hence, we need a technique that is scalabe to complex systems. 
	
\end{enumerate}

 We show that \ac{MILP} satisfies these goals, which is why we choose it in our work. % by providing us generality and scalability. 


\subsection{ Mapping  Mixed-Integer Linear Programming Model to Deep Neural Networks}

DNN architectures are quite complex making it difficult to extract direct mappings from inputs to outputs. 
A \ac{DNN}'s  mathematical structure contains multiple layers, which is the first layer of complexity. The non-linearity introduced due to the activation functions within the network adds another layer of complexity for modeling in \ac{MILP}  that we elaborate below. 
The complexity arises due to 2 factors.
\begin{enumerate}
	\item The mutliple layers in \ac{DNN}.
	\ac{APS} has two layers as explained in Section ~\ref{aps} with $8$ neurons each. 
	\item The non-linearity  of the activation function. 
	Non-linear terms are not allowed in \ac{MILP} modeling \cite{gnonlinearity}
\end{enumerate}

We formalize \ac{DNN} as sets of equations.
We use these equations to build \ac{MILP} models by adding constraints on the variables. 
We handle the non-linear activation function by representing it as a piecewise linear function that can be expressed as a \ac{MILP} model. 
We tackle the non-linearity of activation functions by defining them as piece-wise linear functions, which we demonstrate for the ReLU activation function.
We chose ReLU as it is used in all 3 systems we use for evaluation. 

\subsection{Modeling cost function}
The goal of the cost function is to minimize or maximize the linear constraints that represents the \ac{DNN} architecture and produce an optimization problem. 
We want to minimize the input perturbations and maximize the output deviations in a \ac{DNN} as per our attack model. 
There are two main problems that arise based on our cost function modeling:
\begin{enumerate}
	\item It is possible that no solution exists. 
	This happens if the \ac{MILP} model is infeasible, as defined in ~\ref{background}, and we need to redefine our cost function. 
	\item If the \ac{MILP} model is feasible, there is still the possibility that there is no single optimal solution. 
	This makes our solution space unbounded and can lead to state space explosions. 
\end{enumerate}

We demonstrate how we solve both problems in our experimental evaluation in Chapter ~\ref{evaluation}, scaling our approach. 
We use interval analysis for this purpose, which we discuss in Chapter ~\ref{relusyn}. 


















