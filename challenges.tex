\section{Challenges}
There are 3 challenges in the design of the technique. 
(1) choosing a mathematical abstraction technique that works best for our use case, 
(2) mapping the \ac{DNN}  model to the abstraction, and 
(3) designing the \ac{DNN}  for our use case, namely attack synthesis. 
We elaborate on them below. 


\subsection{Selecting  the automated technique}
The first part is choosing a technique that allows the attacker to model a DNN and find the critical inputs through the modeling. 
There are multiple means of modeling the DNN to synthesize attacks, e.g., SMT solvers, symbolic execution and MILP.
Our goal is twofold: 
\begin{enumerate}
	\item Generality: \ac{APS} has an architecture that %despite being similar to our other evaluated systems 
	is quite different from air control traffic management systems (\ac{ACAS-Xu} and \ac{HCAS}) as observed from Table 6.1.
	This is due to the difference in the application domain and the \ac{DNN} architecture design. 
	\ac{APS} is a medical system that consists of a standard feed-forward network with no exotic features such as normalization 
	unlike \ac{ACAS-Xu}. 
	To accommodate normalization, we have to translate it to a \ac{MILP} model which is not straightforward.  
	Hence, we want an abstraction technique that supports such different architectures. 
	\item Scalability: \ac{APS} consists of 2 hidden layers with 50 neurons each which basically means that there are $2800$ ($50x5 + 50x50 + 50x1$) connections in total.
	On the other hand, \ac{ACAS-Xu} consists of 5 hidden layers with 50 neurons each which means the total connections are $10,500$. 
	Hence, we need a technique that is scalabe to complex systems. 
	
\end{enumerate}

 We show that \ac{MILP} satisfies these goals, which is why we choose it in our work. % by providing us generality and scalability. 


\subsection{ Mapping  Mixed-Integer Linear Programming Model to Deep Neural Networks}

DNN architecture is quite complex as explained in ~\ref{dnncomplexity}, and researchers have used experimental means to understand the security scenarios in the DNN. 
\ac{DNN}s  mathematical structure contains multiple layers, which is the first layer of complexity, and the non-linearity introduced due to the activation functions within the network add another layer of complexity for modeling in \ac{MILP}  that we elaborate below. 
\begin{enumerate}
	\item Modeling each layer as a set of constraints to build a \ac{MILP} model.
	\ac{APS} has two layers as explained in Chapter 3 and has 50 neurons each. 
	\item Non-linear terms are not allowed in \ac{MILP} modeling \cite{gnonlinearity}. Hence, the main challenge is how to model the activation functions since they are non-linear. 
\end{enumerate}

We formalize the \ac{DNN} to represent the entire network as a set of linear constraints. 
We tackle the non-linearity of activation functions by defining them as piece-wise linear functions, 
which we demonstrate for one activation function ReLU.
We chose ReLU because are three evaluation systems have ReLU in their models. 

\subsection{Modeling cost function}
The goal of the cost function is to minimize or maximize the linear constraints that represents the \ac{DNN} architecture. 
However, our requirement is different in the traditional sense of using minimization and maximization, which is usually used in terms of resource utilization ~\cite{7214162}.  
Based on our attack model, we want to minimize the input perturbations, and maximize the output deviations in a \ac{DNN}.
We do so to find the optimal solutions, which we have defined in  ~\ref{milp}
There are two main problems that arise based on our cost function modeling:
\begin{enumerate}
	\item There is a possibility that there are no solutions. 
	This happens if the \ac{MIP} is infeasible as defined in ~\ref{background} and we need to redefine our cost function. 
	\item If the \ac{MIP} is feasible, then there is a possibility that there is no optimal solution since there are infinitely many good cost function values. 
	This makes our solution space unbounded, and can lead to state space explosions. 
\end{enumerate}

We demonstrate how we tackle both the problems and the state space explosion problem in our experimental evaluation, thereby scaling our approach for big sized systems. 
We use interval analysis for this purpose, which we elaborate in Chapter ~\ref{relusyn}. 


















